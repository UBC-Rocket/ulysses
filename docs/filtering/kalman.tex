\documentclass{article}
\usepackage[margin=1in]{geometry} 
\usepackage{amsmath,amsthm,amssymb}
\usepackage{hyperref}

\title{Kalman Filtering \& Sensor Fusion}
\author{UBC Rocket (TVR, Ulysses)}
\date{\today}

\begin{document}

\maketitle % Displays the title, author, and date

\section{Introduction}

The goal is to take the datastream provided from the IMU and through some process determine the current orientation of Ulysses as a quaternion. 
We have data from the IMU's accelerometer and gyroscope, which we can refer to as follows:
$$
\text{Acceleration} =
\begin{pmatrix}
a_x\\
a_y\\
a_z
\end{pmatrix}
\ \ \ \
\text{Angular Velocity} =
\begin{pmatrix}
\omega_x\\
\omega_y\\
\omega_z
\end{pmatrix}
$$
Which we want to use to find and continuously update $q=a_0+a_1i+a_2j+a_3k$. 

\section{Math}
\subsection{Notation}
Denote $q_{n,k}$ as the prediction of $q$ for time $n$ made at time $k$.
Similarly, we define $a_{x|t}$ and $\omega_{x|t}$ as the value of their respective measurements at time $t$.
Our goal, then, is to find $q_{t,t}$ at our current timestep $t$, making use of $q_{t,t-1}$,$q_{t-1,t-1}$, and the measured data.
Note that the initial state is $q=1+0i+0k+0j$. We will update the state via quaternion multiplication.
\subsection{Accelerometer}
When Ulysses is stationary or moving at a constant velocity, we expect the magnitude of its acceleration to be approximately equal to 1 (units in g). 
We can use trigonometry to find the roll/pitch.
\[
\begin{gathered}
\text{Roll}=\theta_x=\text{arctan}2(-a_y, -a_z)
\\\text{Pitch}=\theta_y=\text{arctan}2(a_x, \sqrt{a_y^2 + a_z^2})
\end{gathered}
\]
Each of these corresponds to a quaternion rotation:
$$
\begin{gathered}
q_{\text{roll}}=\cos(0.5\cdot\theta_x)+\sin(0.5\cdot\theta_x)(i+0j+0k)\\
q_{\text{pitch}}=\cos(0.5\cdot\theta_y)+\sin(0.5\cdot\theta_y)(0i+j+0k)
\end{gathered}
$$
(there is a better way to do this)  
To update the given state $q$ with any given quaternion rotation $r$, 
$$
q_{\text{new}}=r\cdot q\cdot r^{-1}
$$
Where $r^{-1}$ is the conjugate of $r$. These angles will not be accurate if Ulysses is experiencing acceleration from other sources. 
We want to fuse this data in order to come to something more accurate regardless of experienced acceleration.

\subsection{Gyroscope}
By integrating over time with respect to angular velocity, we can find the current angles of Ulysses.
$$
q_{t,t}=q_{t-1,t-1}+\dfrac{\Delta t}{2}\cdot w\cdot q_{t-t,t-1}
$$
Where $w$ is the quaternion $w=0+\omega_xi+\omega_yj+\omega_zk$. \href{https://gamedev.stackexchange.com/questions/108920/applying-angular-velocity-to-quaternion}{Find the derevation here}.
We expect the initial values of $\theta_{0,0}$ to be 0, as otherwise have no point of reference to begin our calculations.
The key issue with the gyroscope is that it reports a noise 0.1\textdegree. Over the course of a flight, this uncertainty will compound, resulting in increasingly erroneous data eventually leading to a spectacular crash.
Thus, while the gyroscope data is accurate regardless of acceleration, its noise makes it prone to drift.

\section{Filtering}
With the setups done, we can begin the process of filtering the data.
%    jacobianF = np.array(v[[1.0+(tri[0,0]*tri[1,2]*g[1][0]-tri[0,1]*tri[1,2]*g[2][0])*self.dts, (tri[0,1]/tri[1,0]/tri[1,0]*g[1][0]+tri[0,0]/tri[1,0]/tri[1,0]*g[2][0])*self.dts], 
%                               [-(tri[0,1]*g[1][0]+tri[0,0]*g[2][0])*self.dts, 1.0]])
The jacobian is given By
$$
H=\begin{bmatrix}
    
\end{bmatrix}
$$
\end{document}